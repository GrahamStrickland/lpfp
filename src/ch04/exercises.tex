% Answers for exercises in ch04

\documentclass{article}
\usepackage[utf8]{inputenc}
\usepackage[english]{babel}
\usepackage{amsfonts}
\usepackage{amsmath}
\usepackage{amssymb}

\title{Learn Physics with Functional Programming - Scott N. Walck - Chapter 4: Exercises}
\author{Graham Strickland}

\begin{document}
\maketitle  

\begin{itemize}
\item[4.3]
    For $f(x) = x^3$, we have $f'(x) = 3x^2$, so that the relative error is defined by
    \begin{equation*}
        \begin{split}
            \text{err}(a) &= \left| \frac{f(x + a/2) - f(x - a/2)}{a} - 3x^2 \right| \\
            &= \left| \frac{(x + a/2)^3 - (x - a/2)^3}{a} - 3x^2 \right| \\
            &= \left| \frac{[x^3 + (3x^2a)/2 + (3xa^2)/4 + a^3/8] - [x^3 - (3x^2a)/2 + (3xa^2)/4 - a^3/8]}{a} - 3x^2 \right| \\
            &= \left| \frac{3x^2a + a^3/4 - 3x^2a}{a} \right| \\
            &= \left| \frac{a^3}{4} \right|.
        \end{split}
    \end{equation*}
    \qquad Thus we have an error of 1 percent if 
    \begin{equation*}
        \begin{split}
            \text{err}(a) &= 0.01 \\
            \Leftrightarrow \left|\frac{a^3}{4}\right| &= 0.01 \\
            \Leftrightarrow \frac{a^3}{4} &= \pm 0.01 \\
            \Leftrightarrow a^3 &= \pm 0.04 \\
            \Leftrightarrow a &= \pm {0.04}^{(1/3)},
        \end{split}
    \end{equation*}
    which is valid for $x = 4$ and $x = 0.1$, since $\text{err}(a)$ does not depend on $x$.
\end{itemize}

\end{document}

