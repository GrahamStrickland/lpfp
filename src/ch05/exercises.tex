% Answers for exercises in ch05

\documentclass{article}
\usepackage[utf8]{inputenc}
\usepackage[english]{babel}
\usepackage{amsfonts}
\usepackage{amsmath}
\usepackage{amssymb}
\usepackage{mathtools}

\title{
    Learn Physics with Functional Programming\\---\\Scott N. Walck\\---\\
    Chapter 5: Exercises
}
\author{Graham Strickland}

\begin{document}
\maketitle  

    \begin{itemize}
        \item[5.4] We have the function \verb|range| with the following definition:
        \begin{verbatim}
range :: Int -> [Int]
range x = if x >= 0
          then [0..x]
          else [x..0]
        \end{verbatim}
        \verb|range| returns a list containing all the integers between the argument 
        (inclusive) and 0 in increasing order, i.e, $\text{range}(2) = 0, 1, 2$, 
        $\text{range}(-4) = -4, -3, \ldots, 0$, and $\text{range}(0) = 0$.\par
        \qquad We demonstrate as follows:
        \begin{verbatim}
ghci> range (-4)
[-4,-3,-2,-1,0]
ghci> range 2
[0,1,2]
ghci> range (-4)
[-4,-3,-2,-1,0]
ghci> range 0
[0]
        \end{verbatim}
        \item[5.5] We have the function \verb|null'| with the following definition:
        \begin{verbatim}
import Data.Foldable

null' :: (Foldable t) => t a -> Bool
null' xs = case toList xs of
  [] -> True
  (_ : _) -> False
        \end{verbatim}
        \verb|null'| returns \verb|True| if an argument \verb|t| of type \verb|a| 
        is empty, otherwise \verb|False|. Since we are using the \verb|Foldable| 
        type, we import \verb|Data.Foldable|.\par
        \qquad We demonstrate as follows:
        \begin{verbatim}
ghci> null' []
True
ghci> null' [1, 2, 3]
False
ghci> null' [1..]
False
        \end{verbatim}
        \item[5.6] We have the function \verb|last'| with the following definition
        \begin{verbatim}
import GHC.Stack (HasCallStack)

last' :: HasCallStack => [a] -> a 
last' x = head (reverse x)
        \end{verbatim}
        \verb|last'| returns the last item in an argument with type that implements 
        \verb|HasCallStack|, an error if the argument is empty, or hangs indefinitely
        if the variable has infinite length.\par
        \qquad We demonstrate as follows:
        \begin{verbatim}
ghci> last' [1, 2, 3]
3
ghci> last' ["check", "mate"]
"mate"
ghci> last' []
*** Exception: Prelude.head: empty list
CallStack (from HasCallStack):
  error, called at libraries/base/GHC/List.hs:1646:3 
    in base:GHC.List
  errorEmptyList, called at libraries/base/GHC/List.hs:85:11 
    in base:GHC.List
  badHead, called at libraries/base/GHC/List.hs:81:28 
    in base:GHC.List
  head, called at last.hs:4:11 in main:Main
  last', called at <interactive>:4:1 in interactive:Ghci3
        \end{verbatim}
        \item[5.7] We have the function \verb|palindrome| with the following 
        definition 
        \begin{verbatim}
import Distribution.Simple.Utils

palindrome :: String -> Bool
palindrome s = reverse (lowercase s) == lowercase s 
        \end{verbatim}
        \verb|palindrome| uses the function 
        \verb|Distribution.Simple.Utils.lowercase| to check if the lowercase version 
        of a string is the same as the lowercase version reversed, i.e., is the 
        string a palindrome.\par
        \qquad We demonstrate as follows:
        \begin{verbatim}
ghci> palindrome "Radar"
True
ghci> palindrome "MadamImAdam"
True
ghci> palindrome "racecar"
True
ghci> palindrome "dog"
False
        \end{verbatim}
        \item[5.8] We find the first five elements of the infinite list 
        $[9, 1, \ldots]$ as follows:
        \begin{verbatim}
ghci> take 5 [9,1..]
[9,1,-7,-15,-23]
        \end{verbatim}
        Thus we see that the first five elements are given by
        \[
            [9, 1, \ldots] = [9, 1, -7, -15, -23, \ldots].
        \]
        \item[5.9] We have the function \verb|cycle'| with the following 
        definition 
        \begin{verbatim}
import GHC.Stack (HasCallStack)

cycle' :: forall a. HasCallStack => [a] -> [a]
cycle' xs = concat (repeat xs)
        \end{verbatim}
        \verb|cycle'| repeats an argument which implements \verb|HasCallStack| an
        infinite number of times.\par
        \qquad We demonstrate as follows:
        \begin{verbatim}
ghci> take 10 (cycle' [4,7,8])
[4,7,8,4,7,8,4,7,8,4]
ghci> take 10 (cycle' [1])
[1,1,1,1,1,1,1,1,1,1]
        \end{verbatim}
        \item[5.10]
        \begin{itemize}
            \item[(a)] \verb|["hello",42]| is not a valid expression, since it
            attempts to construct a list from elements of two different types,
            \verb|String| and |Int|.
            \item[(b)] \verb|['h',"ello"]| is not a valid expression, since it
            attempts to construct a list from elements of two different types,
            \verb|Char| and |String|.
            \item[(c)] \verb|['a','b','c']| is a valid expression.
            \item[(d)] \verb|length ['w','h','o']| is a valid expression.
            \item[(e)] \verb|length "hello"| is a valid expression.
            \item[(f)] \verb|reverse| is a valid expression, even though GHCI cannot
            print it.
        \end{itemize}
        \item[5.10] It seems as if an arithmetic sequence will end at the last 
        integer in the sequence before the last element in the constructor if the 
        sequence is an integer sequence.\par
        \qquad If it is a floating point sequence, i.e., one of the elements in the 
        constructor was of floating point type, then the last number in the sequence 
        will be the number in the sequence occurring after the last element in the 
        constructor if that last element is further than the midpoint between two 
        elements in the sequence, otherwise it will be the number occurring before.\par
        We demonstrate as follows:
        \begin{verbatim}
ghci> [0,3..7.5]
[0.0,3.0,6.0,9.0]
ghci> [0,3..7.49]
[0.0,3.0,6.0]
ghci> [0,3..7.499999999]
[0.0,3.0,6.0]
ghci> [0,3..7]
[0,3,6]
ghci> [0,3..8]
[0,3,6]
ghci> [0,3..9]
[0,3,6,9]
        \end{verbatim}
    \end{itemize}
\end{document}
