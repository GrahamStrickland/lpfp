% Answers for exercises in ch05

\documentclass{article}
\usepackage[utf8]{inputenc}
\usepackage[english]{babel}
\usepackage{amsfonts}
\usepackage{amsmath}
\usepackage{amssymb}
\usepackage{mathtools}

\title{
    Learn Physics with Functional Programming\\---\\Scott N. Walck\\---\\
    Chapter 5: Exercises
}
\author{Graham Strickland}

\begin{document}
\maketitle  

    \begin{itemize}
        \item[5.4] We have the function
        \begin{verbatim}
range :: Int -> [Int]
range x = if x >= 0
          then [0..x]
          else [x..0]
        \end{verbatim}
        which returns a list containing all the integers between the argument 
        (inclusive) and 0 in increasing order, i.e, $\text{range}(2) = 0, 1, 2$, 
        $\text{range}(-4) = -4, -3, \ldots, 0$, and $\text{range}(0) = 0$.\par
        \qquad We demonstrate as follows:
        \begin{verbatim}
ghci> range (-4)
[-4,-3,-2,-1,0]
ghci> range 2
[0,1,2]
ghci> range (-4)
[-4,-3,-2,-1,0]
ghci> range 0
[0]
        \end{verbatim}
    \end{itemize}
\end{document}
