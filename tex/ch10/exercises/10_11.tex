% Exercise 10.11

We have the following output using the functions defined in
\path{src/Ch10/NonuniformCircularMotion.hs}

\begin{verbatim}
ghci> radialComponent 2 2
vec (-69.11249319950134) 20.10935225661961 0.0
ghci> radialComponent 2 2
ghci> vNCM 0.01 (2, thetaFunc) 2
11.999999999999744
ghci> magRadialComponent 2 2
71.9786271363377
ghci> squareSpeedDivRad 2 2
71.99999999999693
\end{verbatim}

Thus we can conclude that the radial component of acceleration at $t = 2$s is
given by the vector
\begin{equation*}
    \begin{split}
        \mathbf{a}_r\left(R, \theta(t), t\right) 
        &= \mathbf{a}_r\left(2, \theta(2), 2\right) \\
        &\approx (-69.11249319950134, 20.10935225661961, 0.0),
    \end{split}
\end{equation*}
the speed of the particle at time $t = 2$s is given by 
\begin{equation*}
    \begin{split}
        v_{\text{NCM}}\left(R, \theta(t), t\right)
        &= v_{\text{NCM}}\left(2, \theta(2), 2\right) \\ 
        &\approx 11.999999999999744,
    \end{split}
\end{equation*}
and the magnitude of the radial component at that time by
\begin{equation*}
    \begin{split}
        \lvert \mathbf{a}_r\left(R, \theta(t), t\right) \rvert 
        &= \lvert \mathbf{a}_r\left(2, \theta(2), 2\right) \rvert \\ 
        &\approx 71.99999999999693.
    \end{split}
\end{equation*}
